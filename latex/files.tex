\documentclass[a4paper,11pt]{article}
\usepackage{ulem}
\usepackage{a4wide}
\usepackage[dvipsnames,svgnames]{xcolor}
\usepackage[pdftex]{graphicx}
\title{Introductory Programming in Python: Files}
\usepackage[utf8]{inputenc}
\usepackage{hyperref}
% commands generated by html2latex


\begin{document}

\section{Introductory Programming in Python: Lesson 18
\\    Files for Input and Output}    [\href{random.html}{Prev: Random Numbers}]\nolinebreak[\href{index.html}{Course Outline}]\nolinebreak[\href{regexp.html}{Next: Regular Expressions}]   

\section{Filesystems, Directories, and Files}

The concept of a file is fairly intuitive, but as with all things   programming, intuition is not enough. Let us briefly explore exactly   what a file is. In everyday use, documents, songs, video clips, movies etc are often referred to as files. It is   important to understand that these names are in fact references to the \textit{use} to which a   file is put, rather than the idea of what a file actually is. So what   is a file?  Simply put, a file is an ordered collection of data   associated with a name and location by a filesystem on a device. The device might be a hard drive, a flash disk, a CD-ROM, or   even your cellphone. The same file may be used for many different   things. I can try read my .mp3 file as a text file, for example. It makes very little   sense, but it is possible. Alternatively, one could attempt to play the   contents of a large spreadsheet through one's sound card. Again, it   won't make much sense, and in fact it'll just sound like a burst of static,   but it is possible.\textbf{A file is a name associated with an ordered collection of data on some storage medium}

Another concept that the computer interfaces of the 90s onwards have eroded   significantly is the idea of the filesystem.       Viewed on a physical storage device, we encounter a number of problems dealing directly with   files. This is caused by the way files are stored on the storage device. A single file is not always stored in one big chunk. Rather, it can be \textit{fragmented}. This means that bits of the file are stored in different places. A second problem is that files aren't stored in any particular order. Your word processing documents might be stored right next to, or even mixed with, files belonging to the operating system, your music   collection, or your applications. Clearly we need a way of imposing a   logical structure onto a collection of files. So we introduce a method   of grouping arbitrary files together, namely the   \textbf{directory}, which you may know of as a folder.   Generally, any file can be put into any directory, but cannot exist in   multiple directories at once. Thus we have given files a   \textbf{location}. However, directories can also contain other   directories. This forms an ordered structure of files within   directories, within other directories, within other directories, within \ldots\ and so on.      Where does   it end? It does end, or rather it \textit{starts} somewhere!\textbf{A directory is an arbitrary unordered grouping of files}

Modern GUI (graphical user interface) operating systems make it very easy for us to navigate through folders on our computers, but how does a program see directories and files? Every filesystem has a starting point called the   \textbf{root}. All directories on the filesystem will use this root to describe their location.   In Windows for example, each filesystem is assigned its own root using a   letter from the alphabet, as in 
\texttt{C:$\backslash$}. Note the backslash!   In other operating systems like Linux, the root is simply   called 
\texttt{/}, and other filesystems can be placed inside the   root, much like directories can be placed inside one another. So now we   have a way to specify a particular file exactly, even if two files   might have the same name, no two files can have exactly the same   location. The location of a file is always specified from the root down   through each directory to the file, including the name of the file,   e.g.
\begin{lstlisting}
C:\Documents and Settings\James\Desktop\todo.txt
/home/james/Desktop/todo.txt\end{lstlisting}

Note how in both cases we start with the root, and name each   directory successively, zeroing in on the directory containing the file   of interest. We separate the directory names with 
\texttt{$\backslash$} in the   case of Windows/MS-DOS, or 
\texttt{/} in the case of Linux.   The explicit sequence from root to name is known as the \textbf{full   path} of the file, as we have followed the full path from root   through each directory to the file.

\section{Relative Paths and the Working Directory}

Of course specifying the full path of every file every time we wish   to use it is inconvenient. Many operating systems thus include the idea   of a \textbf{working directory}. Working directories are tied   to login sessions, and are not readily apparent in GUIs. When a user   logs in to a system, their working directory for that login session is   usually set to their home directory. You may have noticed this when you save a document from your word processor: the default directory it saves to is the working directory. Since programs can't use the GUI to change the working directory we need another method. Various OS commands can change or   print out the working directory. 
\texttt{cd} is used to change the   working directory to another one (it stands for 'change directory').  Whenever a file or directory is   specified, and the specification is not a full path, the file is   considered relative to the current working directory. For example,   specifying only a name for a file, implies the file we are looking for   is in the working directory. When program runs, it inherits the working   directory from the login session from which it is run. In the example   below, the full path to the working directory is displayed in the   prompt.
\begin{lstlisting}
/home/james $ ls
todo.txt
/home/james $ cd /home
/home $ ls
james
/home $ cat james/todo.txt
I have nothing to do!
/home $ \end{lstlisting}

Note how the final command specifies an incomplete path   'james/todo.txt', and not the full path. Because a full path was not   specified, the working directory ('/home' at the time) is added to the start of   the name, yielding '/home/james/todo.txt'. Thus we are able to specify   files and directories lower down in the directory hierarchy in a   relative manner.

Of course if a file is in a directory somewhere above or rather   outside the working directory, it would seem we must still use the full   path to specify it, but there are a few shortcuts we can take.
\begin{itemize}
	\item 
\texttt{./} indicates the working directory, not much help to us really\ldots
	\item 
\texttt{../} indicates the directory above the directory specified in the path so far.
\end{itemize}

For example, if the working directory were '/home/james/IntroPython'
\begin{itemize}
	\item 
\texttt{index.html} would have the full path 
\texttt{/home/james/IntroPython/index.html}
	\item 
\texttt{./index.html} would have the full path 
\texttt{/home/james/IntroPython/index.html}
	\item 
\texttt{data/testinput.txt} would have the full path 
\texttt{/home/james/IntroPython/data/testinput.txt}
	\item 
\texttt{../todo.txt} would have the full path 
\texttt{/home/james/todo.txt}
	\item 
\texttt{../amusement/phdcomics.tar.gz} would have the full path 
\texttt{/home/james/amusement/phdcomics.tar.gz}
	\item 
\texttt{../../../bin/ls} would have the full path 
\texttt{/bin/ls}
	\item 
\texttt{/home/james/../../bin/ls} would have the full path 
\texttt{/bin/ls}
\end{itemize}

\section{Files for Input and Output}

And all this has been leading up to the idea that programs need not   limit themselves to key strokes and the screen as input and output   methods respectively. Files can be used as both input and output. Files   allow us to conveniently input large quantities of data, and similarly   output large quantities of data for later review. When we want to work   with files, we must clearly be able to specify the file we wish to work   with, using it's location, and name, i.e. a valid full or relative   path.

\section{Opening Files}

Opening a file in Python is simple. We use the \textbf{open} function.   The open function returns a file object, given a path to a file in the form of string, and a string   specifying whether to open the file for reading or writing.

All files in Python are treated as text files, meaning they are   broken up into units called lines. However, there is also a concept   called a \textbf{file pointer}. A file pointer is like a cursor   in a word processor, which sits between characters in a file. When we   read from a file, we read from the file pointer onwards to the right.   Similarly all writing done to the file will be from the file pointer   onwards, overwriting if the file already contained data to the right of   the file pointer, otherwise extending the size of the file as   necessary. Both reading and writing a file, reposition the file pointer   at the end of the sequence read or written.
\begin{lstlisting}

>>> f = open("input.txt","r")
>>> f
$<$open file 'input.txt', mode 'r' at 0xb7dacentered$>$
>>>
\end{lstlisting}

Here we see the open function in action. It is most commonly used as   the expression of an assignment statement, but obviously, being a   function, can be used anywhere an expression is valid. The open   function takes two parameters, firstly, the path of the file to open,   and secondly, a string specifying the mode (read, write, or append) in   which to open the file. Valid strings for the mode are
\begin{itemize}
	\item "r" opens the file in read only mode. The file must exist prior    to opening. A plus suffix ("r+") means the file is opened for both    reading and writing. In either case the file is opened with the    file pointer at the beginning of the file.
	\item "w" opens the file in write only mode. The file will be    overwritten if it already exists, otherwise it will be created. A    plus suffix ("w+") means the file is opened for both reading and    writing, but the file is still truncated to zero bytes if it exists    already. Obviously this means the file pointer starts at the    beginning of the file. Writing beyond the end of a file simply    enlarges the file to accommodate whatever is written.
	\item "a" opens the file in append mode. The file can only be written    to, and the file pointer starts at the \textbf{end} of the    file, meaning all data subsequently written will be added to the    end of the file. If the file doesn't already exist, it will be    created. A plus suffix ("a+") means the file is opened for both    reading and writing, however the file pointer is positioned at the    \textbf{beginning} of the file, meaning "a+" and "r+" are    essentially equivalent.
\end{itemize}

\section{Reading From Files}

Once we have an open file object, we can use its methods to both   read from and write to the file it represents. When a file is opened   for reading, the file pointer is positioned at the beginning of the   file (position 0) which is just before the first character in the file.   File objects provide a variety of methods to read from files\ldots
\begin{itemize}
	\item 
\texttt{$<$file object$>$.read($<$count$>$)} reads    'count' characters from the file object starting at the file    pointer, and returns the read characters as a string. If less than    'count' characters remain to be read, then a string is still    returned, but it will be shorter than count characters. An empty    string is returned if the file pointer is at the end of the    file.
	\item 
\texttt{$<$file object$>$.readline()} reads from the file    pointer onwards up to and including the first newline ('$\backslash$n')    character, and returns the read characters as a string. An empty    string is returned if the file pointer is at the end of the    file.
	\item 
\texttt{$<$file object$>$.readlines()} reads from the    file pointer until the end of file, returning a list of lines, each    containing the trailing newline characters, as strings. An empty    list is returned if the file pointer is already at the end of the    file.
\end{itemize}

Using the following file as an example
\begin{lstlisting}
This is a simple file
Containing only three lines
of text\end{lstlisting}

We can demonstrate the use of the various file reading methods.
\begin{lstlisting}
>>> f.read(3)
'Thi'
>>> f.readline()
's is a simple file\n'
>>> f.readlines()
['Containing only three lines\n', 'of text\n']
>>>\end{lstlisting}

Note how using the simple 'read' method, we get only three   characters (being the count we specified), and the 'readline' method   continues from where 'read' finished. This illustrates the file pointer   in action. The file pointer is position between the 'i' and 's' of   'This' on the first line of the file after the 'read' method is   executed. After the 'readline' it is between the end of line one and   the first character of line two ('C'). Thus, 'readlines' has two   complete lines left to read when it is called.

Often a more useful way to read the lines of a file in sequence, is   the for loop construct over a file. When used in a for loop a file   object acts as a sequence of lines, as in
\begin{lstlisting}
>>> f = open("input.txt","r")
>>> for line in f:
...		 print line.rstrip()
... 
This is a simple file
Containing only three lines
of text
>>>\end{lstlisting}

\section{Writing To Files}

Writing files uses methods very similar to those used to read from   files, except that writing is often buffered in memory, meaning the   file on disk is only actually updated when newlines are written, the   buffer is explicitly flushed, or the file object is closed.
\begin{itemize}
	\item 
\texttt{$<$file object$>$.write($<$string$>$)} writes    the contents of 'string' to the file, starting at the file pointer    and overwriting data in the file, or enlarging the file as    necessary. Note that no newline is added, so multiple successive    write calls without any newlines in their respective strings,    produce only one line of text in the file.
	\item 
\texttt{$<$file object$>$.writelines($<$list$>$)}    writes the elements of the list, which must all be strings, in    order to the file. Newlines are not added, so if the strings have    no newlines, only one line of output will be written.
\end{itemize}

As an example, let's create a new file, and write some text out to   it.
\begin{lstlisting}
>>> w = open("newfile.txt","w")
>>> w.write("This is the first line of text ")
>>> w.write("This is still on the first line\n")
>>> w.writelines(["The second line\n", "The Salmon Mousse"])
>>> w.close()
>>> \end{lstlisting}

Looking at 'newfile.txt', we see
\begin{lstlisting}
This is the first line of text This is still on the first line
The second line
The Salmon Mousse\end{lstlisting}

We see from the first two calls to 'write' that we have to supply   our own newline characters to force line breaks in the output. And   finally that closing our files is a good idea when they are opened for   writing. Technically, Python's garbage collector usually takes care of   this for us, closing file objects before they are collected, but it is   considered good practice to explicitly close files.
\begin{itemize}
	\item 
\texttt{$<$file object$>$.close()} closes the file,    disallowing further read or write operation on the file. Any    pending written data in the file buffer is flushed to disk.    Technically, close is called automatically when a file object    variable is garbage collected, but it is good practice to    explicitly close the files your program opens, as the operating    system has a limit to the number of files that may be open at one    time.
\end{itemize}

\section{Moving the File Pointer Manually}

Occasionally we may want to move the file pointer manually, for   example when reading a file of a format that allows us to skip or   ignore large sections to get to the part of the file we want. Python   provides two methods of file objects to do this, the first to tell us   where the pointer is currently, and the second to move it. Both treat   the file as a one dimensional stream of characters, much like a string.   The file pointer is an integer index into this 'string' specifying the   first character that would be read or written in the next   operation.
\begin{itemize}
	\item 
\texttt{$<$file object$>$.tell()} returns the index of    the file pointer for the file.
	\item 
\texttt{$<$file object$>$.seek($<$position$>$[,     $<$whence$>$])} moves the file pointer to position     relative to a position indicated by whence. Whence can take on     one of three values, defaulting to 0, each of which mean      
\begin{enumerate}
	\item Set position relative to the beginning of the      file.
	\item Set position relative to the current position, i.e.      negative positions will be before the current position, and      positive positions will be after it.
	\item Set position relative to the end of the file, i.e. only      negative positions will give us a position inside the      file.
\end{enumerate}
\end{itemize}

\section{Truncating a File}

Now that we can move the file pointer manually, we can do some funky   things. We could for example seek to somewhere in the middle of a file,   and overwrite the data there, without affecting the rest of the file.   We might even want to remove information from a file, where we seek to   the start of the information, and overwrite it with\ldots\ well, spaces. Oh   dear, we have a problem here! We can't actually \textit{remove}   information from a file, or can we? The truth is that we can only   shorten the length of file, more formally known as truncating the file.   What this means is that to delete information contained in the file   starting at some position (x) up to some position (y), we must read   everything from y to the end of the file, and then seek back to x, and   write what we read. Finally we truncate the file.
\begin{itemize}
	\item 
\texttt{$<$file object$>$.truncate([size])} truncates the    file to be a maximum of size bytes in length. If size is not given,    the file is truncated at the current file pointer position.
\end{itemize}    [\href{random.html}{Prev: Random Numbers}]\nolinebreak[\href{index.html}{Course Outline}]\nolinebreak[\href{regexp.html}{Next: Regular Expressions}]      Copyright \copyright James Dominy 2007-2008; Released under the \href{http://www.gnu.org/copyleft/fdl.html}{GNU Free Documentation License}
\\\href{intropython.tar.gz}{Download the tarball}

\end{document}
