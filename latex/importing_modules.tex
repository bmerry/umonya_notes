\section{The \texttt{import} Statement}

We have already learnt the basics of programming. We have all the   tools we will ever need to solve any problem we can solve in our heads.   But many times we will find ourselves reinventing the wheel. Either a   wheel we've made ourselves, e.g. writing the same piece of code over   and over again in different programs, or even a wheel someone else has   made, e.g. solving a problem someone else has already solved. The point   is the code has already been written, and what we want is a convenient   way to package it so that we can use it without rewriting it. And thus   the \textit{module} was born.

Modules allow us to write code, and separate it out from other code   into a different file, known as a `module'. We can use the \texttt{import}   statement to include the code of another file, either in the   current directory, or in python's \textit{search path}, in   place. What this means is that the file is loaded and run as if its   code had been in the place of the \texttt{import} statement. As such, any   statements in the file are executed, including function definitions,   class definitions, assignments, etc.
\begin{lstlisting}
import <module_name>
\end{lstlisting}

\texttt{module\_name} is the name of a Python file without the `.py'   extension. Any Python program can be imported as a module, although   most often modules contain function definitions and a minimum of   initialisation code, allowing us to keep commonly used function   definitions in a place where they can be reused without rewriting them,   and more importantly collecting function definitions that are related,   e.g. all database functions, together and separate from our main   code.

\section{Namespaces}

If we have a Python file which we wish to import, that has a global   variable \texttt{i}, into our main program, also with a global variable \texttt{i},   there could be some confusion. The module's \texttt{i} variable is in the   global scope of the module, but where is it in the scope of the program   as a whole? Python solves this problem with the introduction of   namespaces. Formally, a module when imported introduces its own scope   block, called the \textit{module scope}. \textit{It is a global scope in its own   right}.  Variables within a modules global scope are forcibly   kept separate from their super-programs, or super-modules (i.e. those   programs or modules that import them). The global statement does not   allow variables references or assignment to cross module scope   boundaries. Instead all names (variables, functions, or classes) are   created within a module's `namespace', meaning they must be explicitly   referenced from without the module by first naming the module itself,   as in
\begin{lstlisting}
<module_name>.<symbol_name>
\end{lstlisting}

Where \texttt{symbol\_name} is the name of a variable assigned a value, a function   defined, or of a class defined.
\begin{lstlisting}
#my_module.py

i = 2
\end{lstlisting}
\begin{lstlisting}
#main.py
i = 1

import my_module.py
print i
print my_module.i

print "---"

my_module.i = 3
print i
print my_module.i
\end{lstlisting}

Running main.py produces the following output
\begin{lstlisting}
1
2
---
1
3
\end{lstlisting}

Despite the fact that \texttt{i} is assigned the value of 2 in my\_module.py   \textit{after} it is assigned the value of 1 in main.py, since   the \texttt{import} statement executes that assignment after the initial   assignment, the value of \texttt{i} in main.py is not changed. As we can see,   the two \texttt{i}'s are kept completely separate. From main.py, if we wish to   reference the \texttt{i} in my\_module.py, we must do explicitly, using the   module name, 
\texttt{my\_module.i}. We also see that we can assign a   value to a variable in a module's namespace, using the same syntax.

\section{The \texttt{dir} function}
\begin{lstlisting}
dir(<object>)
\end{lstlisting}

The \texttt{dir} function is a built in function that takes one parameter,   and lists all the attributes of the object given. Methods, properties,   and variables are all considered attributes. If this isn't making sense   to you yet, don't worry, we will cover it all in the sections on object   oriented programming. What we need to know now, is that although the   \texttt{dir} function is not very helpful in a program, it is \textit{incredibly   useful} when using the Python interactive interpreter. As   modules are objects in python, in fact everything is an object, we can   call \texttt{dir} on a module to obtain a list of everything in the module that   can be referenced with the dot notation   (
\texttt{<module\_name>.<attribute>}).
\begin{lstlisting}
>>> import my_module
>>> dir(my_module)
[`__builtins__', `__doc__', `__file__', `__name__', `i']
>>>
\end{lstlisting}

In the example, we import my\_module.py (from above), and run \texttt{dir}   on it. We are returned a list containing some stuff we expect, namely   the \texttt{i}, which is a variable assigned a value within the module   my\_module, and some stuff we don't expect; 
\texttt{[`\_\_builtins\_\_',   `\_\_doc\_\_', `\_\_file\_\_', `\_\_name\_\_']}. Short of delving into both   object oriented programming and exception handling all at once, these   elements cannot be easily explained now, and are not especially   important to us.

\section{The \texttt{locals} and \texttt{globals} Functions}

Of more direct use to use whilst programming, are the \texttt{locals} and   \texttt{globals} functions. Both take no parameters, and both return   dictionaries. 
\texttt{globals()} returns a dictionary of objects   available in the global scope, whilst 
\texttt{locals()} does the   same but for the local scope. The practical use of this is we can make   even the names of our variables dynamic, i.e. the ability to use a   variable (\texttt{a}) to store another variable's (\texttt{b}) name, by using \texttt{a} as   a key into the dictionary returned by either locals, or globals.
\begin{lstlisting}
#!/usr/bin/python

#ask the user for a variable name
vname = input("Enter a variable name: ")

#ask the user for a value
value = input("Enter a value for %s: "%vname)
locals()[vname] = value

print "vname: ", vname
print "locals()[vname]: ", locals()[vname]
print "value: ", value

#assume the user inputted bob
print "bob: ", bob
\end{lstlisting}

Outputs the following when `bob' is entered as a variable name, and   4 as a value. Note how `bob' becomes available as a normal variable   name, as seen in the last line, once it has been added to the \texttt{locals()}   dictionary.
\begin{lstlisting}
Enter a variable name: bob
Enter a value for bob: 4
vname:	bob
locals()[vname]:	4
value:	4
bob: 4
\end{lstlisting}

\section{An Overview of Selected Standard Modules}

Ultimately, the true use of modules, is to get other people to do   the work for us. We are programmers, and thus laziness a virtue, after   all. The comprehensive list of standard Python modules, i.e. modules   that come with a standard install of the latest version, can be found at \href{http://docs.python.org/modindex.html}{http://docs.python.org/modindex.html}. Of particular   interest to us, by virtue of their usage most common, are
\begin{description}
	\item[math] Provides various non-basic mathematical functions, \textit{inter    alia}: trigonometric functions, hyperbolic functions, exponent    and logarithm functions, and mathematical constants (e.g. pi and    e)
	\item[random] Provides various methods of choosing random numbers, or making    random choices from sequences, using a variety of statistical    distributions.
	\item[os] Provides multitudinous operating system related functions,    primarily file and process management functions.
	\item[sys] Provides a large number of variables and functions dealing with    the internals of the Python interpreter itself, and the environment    in which it was invoked. Specific uses of various sys variables are    discussed shortly.
	\item[time] Provides various functions for obtaining the current time, and    manipulating variables that store values representing time.
\end{description}

\section{The \texttt{sys} Module}

The \texttt{sys} module is quite extensive, but by far the most useful   aspects of it are access to what are known as the \textit{standard   streams}, and the processes command line arguments. When a   program is run from the command line, for example Python, we can pass   the program itself parameters, as if it were a function. In operating   system shell speak they are known as arguments. We pass an argument to   Python to tell it what Python file to run,
\begin{lstlisting}
sirlark@hephaestus ~ $ python my_file.py
\end{lstlisting}

We can actually pass any number of arguments to a program we run   from the command line, each separated by some amount of white space. If   white space is enclosed in quotes of some kind, it does not separate   arguments, but rather is included within one command line argument. In   Python's case, arguments that come after the specification of the   filename to run are passed through to the program being run as its own   command line arguments.    
\begin{lstlisting}
{sirlark@hephaestus ~ $ python my_prog.py somefile.fasta 7 "ACCTGT AGTCA"
\end{lstlisting}

In the example above, the program my\_prog.py receives three command   line arguments, namely \texttt{somefile.fasta}, \texttt{7}, and \texttt{ACCTGT AGTCA}. Note   the space within the sequence snippet, and the fact that the 7 is in   quotes and thus a string.

So now we have a really convenient way of passing small amounts of   input into our programs, instead of prompting the user with input   calls all the time. Command line arguments are especially useful for   the purposes of specifying filenames and options that influence how our   program processes data, and preferable to input statements for two   reasons. First, when entering filenames, command line arguments allow   the user to use shell expansions and tab completions. Second, our   program can be run without user intervention if it's likely to take a   long time to run, e.g. the user can specify a number of files to   process on the command line (probably using a `*.fasta' or similar) and   walk away. Assuming each file would take 20 or more minutes to process,   if we were using input statements to prompt for the next file, if   the user came back the next morning, our program would have processed   the first file, and be waiting for the entry of the second filename.   Command line arguments avoid this issue neatly.

But how, in Python, do we access what command line arguments have   been given to our program? Well, the \texttt{sys} module provides a list called   `argv' (for argument vector), which contains as its elements the   command line arguments passed to our program, in order. Examine the   following simple program. Run it a few times providing different   command line arguments each time, and test it with quoted arguments   containing spaces or tabs.
\begin{lstlisting}
#!/usr/bin/python
#commargs.py

#import the sys module
import sys

#print out our command line arguments
print sys.argv
\end{lstlisting}
\begin{lstlisting}
nyx@nyx-desktop:~/umonya_notes$ python commargs.py Hello
[`commargs.py', `Hello']
nyx@nyx-desktop:~/umonya_notes$ python commargs.py Hello There
[`commargs.py', `Hello', `There']
nyx@nyx-desktop:~/umonya_notes$ python commargs.py "Hello There"
[`commargs.py', `Hello There']
nyx@nyx-desktop:~/umonya_notes$ python commargs.py "Hello There" 47
[`commargs.py', `Hello There', `47']
nyx@nyx-desktop:~/umonya_notes$
\end{lstlisting}

Note how the first element of 
\texttt{sys.argv} is always the   name of our Python program. Where after the arguments are those we gave   on the command line. Also note how all the elements, even the 47, are   in fact strings.

\section{The Standard Streams}

Whenever a Python program is run, it opens three files   automatically, called standard input (\texttt{stdin}), standard output (\texttt{stdout}),   and standard error (\texttt{stderr}). Generally these files represent keyboard   input, screen output, and error output respectively, but they can be   \textit{redirected} causing the input to our program to come   from the output of another, for example, as when used in a shell pipe   (`|'). Similarly our programs output could be redirected to a file, or   piped into the input of another process. However, all these effects are   transparent to us... the \texttt{print} statement actually writes to \texttt{stdout},   the input function reads from \texttt{stdin}. If we want to write to \texttt{stderr},   however, we must do so explicitly. The three standard streams are all   accessible via the \texttt{sys} module as
\begin{itemize}
	\item 
\texttt{sys.stdin}: standard input
	\item 
\texttt{sys.stdout}: standard output
	\item 
\texttt{sys.stderr}: standard error. This stream exists to    differentiate error output from normal output, so that we can use    the shell to split off errors from processes executed in large    batches, and not have to deal with all the normal output they    produce. This stream is usually outputted to the screen, unless it    has been redirected, which means that programs can generally issue    error messages to the screen even if their normal output has been    redirected.
\end{itemize}

Each of these are file objects, so \texttt{sys.stdin}, which is opened only   for reading, has the usual 
\texttt{.read}, 
\texttt{.readline},   and 
\texttt{.readlines} methods, whilst the other two, \texttt{stdout} and   \texttt{stderr}, being opened only for writing have the 
\texttt{.write} and   
\texttt{.writelines} methods available. Here's an example of how to   use \texttt{stdout} and \texttt{stderr} to split output effectively.
\begin{lstlisting}
#!/usr/bin/python

import sys

#get a line of input from the user
print "Please enter a phrase: "	#this gets sent to stdout
phrase = sys.stdin.readline()	#this gets read from stdin

if phrase.isdigits():
    sys.stderr.write("Phrase entered was a number\n")	#this gets sent to stderr
else:
    sys.stdout.write("Phrase contains %i `a' characters\n"%phrase.count(`a'))
\end{lstlisting}

There's one more feature of the \texttt{print} statement that now becomes important, namely redirection. We can specify a standard stream (or in fact any file object open for writing) to print to, instead of standard output. The syntax is:
\begin{lstlisting}
print >> <file object>, <expression>[, <expression>...]
\end{lstlisting}

So the above example could be redone as follows:
\begin{lstlisting}
#!/usr/bin/python

import sys

#get a line of input from the user
print "Please enter a phrase: "	#this gets sent to stdout
phrase = sys.stdin.readline()	#this gets read from stdin

if phrase.isdigits():
    print >> sys.stderr, "Phrase entered was a number"	#this gets sent to stderr
else:
    print "Phrase contains %i `a' characters"%phrase.count(`a'))
\end{lstlisting}

\section{Exercises}
\begin{enumerate}
	\item What is redirection, and what are it's effects?
	\item When, and why, would you want to output to \texttt{stderr}?
	\item When, and why, are command line arguments preferable to taking    input from the keyboard?
	\item Can you think of uses for 
\texttt{sys.argv[0]}? What are they?
\end{enumerate}
