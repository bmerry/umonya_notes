\documentclass[a4paper,11pt]{article}
\usepackage{ulem}
\usepackage{a4wide}
\usepackage[dvipsnames,svgnames]{xcolor}
\usepackage[pdftex]{graphicx}
\title{Introductory Programming in Python: Exercises}
\usepackage[utf8]{inputenc}
\usepackage{hyperref}
% commands generated by html2latex


\begin{document}

\section{Introductory Programming in Python
\\   Exercises}    [\href{index.html}{Course Outline}]
\\    [\href{solutions}{Solutions}]   

\subsection{Lesson 1}
%  Basic Concepts 

\begin{enumerate}
	\item What is a program?
	\item What is the difference between an expression and a    statement?
\end{enumerate}

\subsection{Lesson 2}
%  Invocation 


Consider the following lines code...
\begin{lstlisting}
a = 9
b = 3
a/b\end{lstlisting}
\begin{enumerate}
	\item What will the output be if these lines are entered into the     python interactive interpreter?
	\item What will the output be if I run these lines from a     file/script?
	\item What changes need to be made to produce output when I run     these lines from a file/script?
\end{enumerate}

\subsection{Lesson 3}
%  Basic Output 


Consider the following code...
\begin{lstlisting}
print "MUCH madness is divinest sense,"
print "To a discerning eye;"
print "Much sense the starkest madness."
print "???T is the majority", "In this, as all, prevails."
print "Assent, and you are sane;"
print "Demur,???you ???re straightway dangerous",
print
print "And handled with a chain."\end{lstlisting}
\begin{enumerate}
	\item What does the print statement do generally?
	\item What special rules does the print statement adhere to regarding    trailing spaces and newlines?
	\item What output, exactly, does the above code produce. Indicate    space with underscores
	\item Write a program that outputs favourite piece of poetry or other prose.
	\item Write a program that outputs your name, age, and height in    metres in the following format. Make sure age is an integer, and    height is a float, and not simply part of your string.
\\
\texttt{My name is James, I am 28 years old and 1.78 metres    tall}
	\item Explain three possible ways to print a string containing an    apostrophe, for example the string
\\
\texttt{The cat's    mat}.
\end{enumerate}

\subsection{Lesson 4}
%  Program state 

\begin{enumerate}
	\item What does the assignment statement do?
	\item What are the five basic variable types in python?
	\item If the integer variable \textit{i} has the value 7, then what is    the value of the expression 
\texttt{7/4}. Why?
	\item How does one calculate the remainder (modulo) of an integer when    divided by another number?
	\item If \textit{s} is a sting variable with the value "Harry's Hippie    Hoedown", then what is the value of
\\
\texttt{s + ": tickets only    \$5"}
	\item If \textit{s} is a sting variable with the value "Harry's Hippie    Hoedown", then what is the value of
\\
\texttt{s + ": tickets only    \$" + "5"*3}
	\item What is the value of 
\texttt{"ABBA was a Swedish band popular during the    80's"[0:4]}?
	\item What is the value of 
\texttt{"ABBA was a Swedish band popular during the    80's"[-15:-7]}?
	\item If the string variable \textit{s} has the value "ABBA was a    Swedish band popular during the 80s", then what is the value of 
\\
\texttt{"BAAB"+s[4:11]+"Danish"+s[19:25]+"un"+s[25:04]+"90's"}
\end{enumerate}

\subsection{Lesson 5}
%  basic input 

\begin{enumerate}
	\item If there is a function named \textit{my\_function}, how do I call    it?
	\item Are functions expressions or statements? What about when they    are called?
	\item Write a program that asks the user to enter two numbers, and    prints the sum of those two numbers.
	\item Write a program that asks the user to enter two numbers, and    prints the difference of those two numbers.
	\item Write a program that asks the user to enter two numbers, and    prints the product of those two numbers. Are you sensing a pattern    here?
	\item Write a program that asks the user for some text, and a number.    The program prints out the text a number of times equal to the    number entered, without line breaks or spaces in between each    repetition.
\end{enumerate}

\subsection{Lesson 6}
%  conditionals 

\begin{enumerate}
	\item Write a program that asks the user to enter two numbers. If the    second number is not zero, print the quotient of the two numbers,    otherwise print a message to the effect of not being able to divide    by zero.
	\item Write a program that asks the user to enter 5 numbers, and    prints the smallest number entered.
	\item Write a program that asks the user to enter 5 numbers, and    prints the largest number entered.
	\item Write a program that asks the user to enter 5 numbers, and    prints their average.
	\item Write a program that asks the user to enter a number from 1 to    12 and prints out the name of the corresponding month.
	\item Write a program that asks the user to enter a four digit year,    and prints out whether that year is a leap year or not.
	\item Write a program that asks the user to enter 3 names, then    outputs them sorted alphabetically.
	\item Write a program that asks the user to enter a number, then    prints out all the numbers from 1 to 10 by which the entered number    is divisible.
\end{enumerate}

\subsection{Lesson 7}
%  conditional loops 

\begin{enumerate}
	\item Write a program that asks the user to enter a sequence of    numbers, ending with a blank line. Print out the smallest of those    numbers.
	\item Write a program that asks the user to enter a sequence of    numbers, ending with a blank line. Print out the average of those    numbers.
	\item Modify your solution to exercise 36 so that instead of printing    the average and then terminating when a blank line is entered, it    prints the average and then asks if the user wants to repeat the    process. If the user answers 'y' the process is repeated, otherwise    the program terminates.
\end{enumerate}

\subsection{Lesson 8}
%  sequential loops 

\begin{enumerate}
	\item Write a program that asks the user for the height of a triangle.    If a blank line is entered, the program finishes, otherwise it    prints out a right handed triangle, with the right angle on the    bottom right, made of asterisks ('*') of a height equal to the    number entered. Example input/output follows ...      
\begin{lstlisting}

Enter triangle height: 3
*
**
***
Enter triangle height: 5
*
**
***
****
*****
Enter triangle height: 
\end{lstlisting}
	\item Same as above, except now with the right angle in the top right! Example ...      
\begin{lstlisting}

Enter triangle height: 3
***
 **
  *
Enter triangle height: 5
*****
 ****
  ***
   **
    *
Enter triangle height: 
\end{lstlisting}
\end{enumerate}

\subsection{Lesson 9}
%  lists 

\begin{enumerate}
	\item Write a program that asks the user to enter a sequence of up to    5 x:y coordinates with both x and y in the range 0 to 4, ending    their sequence entry by providing a blank line for the x    coordinate. Then display a five by five grid of '\#' characters,    with the points in the grid entered by the user left blank. Assume    x increases from left to right, and y increases from top to bottom.    Example input/output is given ...          
\begin{lstlisting}
Coordinates range from 0 to 4!
Please enter pair of coordinates (x:y), leave x blank to terminate sequence.
X$>$ 3
Y$>$ 3
Please enter pair of coordinates (x:y), leave x blank to terminate sequence.
X$>$ 4
Y$>$ 1
Please enter pair of coordinates (x:y), leave x blank to terminate sequence.
X$>$ 1
Y$>$ 4
Please enter pair of coordinates (x:y), leave x blank to terminate sequence.
X$>$
#####
#### 
#####
### #
# ###
\end{lstlisting}
\end{enumerate}

\subsection{Lesson 10}
%  tuples 

\begin{enumerate}
	\item       Write a function that returns the sum of two vectors, where the     sum of two vectors is a vector of the same dimension (i.e. has     the same number of elements) as both the original vectors (i.e.     original vectors must be of equal dimension, otherwise return     None), and has each element equal to the sum of its respective     elements from the original two vectors.
\\
\\
\texttt{[x, y, z] + [a, b, c] = [x+a, y+b, z+c]}
\\
\\     Then ask the user to enter the direction and speed (in the form     of a 2 dimensional vector) of the wind, and again of an object     travelling in a two dimensional space. Your program should     output the vector representing the resultant direction vector of     the object.
\\
\\\href{http://www.physicsclassroom.com/Class/vectors/U3L3a.html}{Some High School Physics Background}
\\\href{http://kipper.crk.umn.edu/physics/1012/xtr/vectors.html}{More Complex, but More Complete}
\end{enumerate}

\subsection{Lesson 11}
%  dictionaries 


\subsection{Lesson 12}
%  strings in depth 


\subsection{Lesson 13}
%  functions 


\subsection{Lesson 14}
%  variable scope 


\subsection{Lesson 15}
%  standard modules (math, sys, os) 


\subsection{Lesson 15}
%  standard modules (math, sys, os) 


\subsection{Lesson 16}
%  command line arguments 


\subsection{Lesson 17}
%  random numbers 


\subsection{Lesson 18}
%  files 


\subsection{Lesson 19}
%  regular expressions 


\subsection{Lesson 20}
%  parsing 


\subsection{Lesson 21}
%  os functionality 


\subsection{Lesson 22}
%  dates and times 


\subsection{Lesson 23}
%  writing error messages 


\subsection{Lesson 24}
%  understanding error messages 


\subsection{Lesson 25}
%  exceptions 


\subsection{Lesson 26}
%  debugging 


\subsection{Lesson 27}
%  recursion 


\subsection{Lesson 28}
%  classes 


\subsection{Lesson 29}
%  queues 


\subsection{Lesson 30}
%  stacks 


\subsection{Lesson 31}
%  trees 


\subsection{Lesson 32}
%  database theory 


\subsection{Lesson 33}
%  relational databases and SQL 


\subsection{Lesson 34}
%  using databases in Python 


\subsection{Lesson 35}
%  parallel processing 


\subsection{Lesson 36}
%  event oreinted programming 


\subsection{Lesson 37}
%  gui programming 


\subsection{Lesson 38}
%  web programming/services 

\begin{enumerate}
	\item A small arts and crafts store owner in the middle of the Karoo    has recently upgraded to a computerised point of sale system, and    wants to do the same for his guest book. Customers have previously    left their names a small paragraph of comment in the book. The owner    would like his customers to be able to walk up to a computer near    the exit, type in their names, and enter a brief comment. He's only    interested in a customer's most recent comments, and doesn't want    store old comments. So repeat customer's must be able to update    their previous comments. When a repeat customer types in their name,    their previous comment is displayed back to them, and they are    afforded the opportunity to enter a new comment. Should they enter a    blank line instead of a comment, their previous comment is    preserved. Also, if instead of a customer name the special command    'quit' is entered, the program exits. Similarly the command    'showcomments' causes all customers' names to be displayed, followed    by their comments slightly indented. Customer's must be able to    enter their names in any case.
	\item Extend your solution to the previous problem, by allowing    customers to enter multi-line comments, and to terminate their    comments by entering a blank line. If the comment is entirely blank,    i.e. the first line is blank, then it does not overwrite the former    comment if any. Also, ensure that when the comments are outputted    back, either because of the 'showcomments' command, or a repeat    customer entering their name, that the line width of the outputted    comments does not exceed 60 characters, nor break a word in two,    i.e. lines are only broken on white space.
\end{enumerate}

\subsection{Lesson 12}
\begin{enumerate}
	\item Write a program to display the contents of a file as is, i.e.    the whitespace and line breaks outputted must be identical to that    in the file. Ask the user which file to display.
	\item Write a program that asks the user for the name of a file, asks    for the name of another file, then outputs the contents of the    first file into the second file, essentially copying it.
	\item Convert the previous program into a function, and allow the    user to continue entering filenames for source and destination    files, until they enter a blank line for the source file.
	\item Instead of copying a file as is, write a new function that    coverts all letters in the source file to uppercase letters when    written to the destination file.
	\item Write a program that reads in a FastA file of the user's    choice, and outputs a list of the 'titles' contained therein. You    can obtain a sample FastA file \href{data/brca.fasta}{here}.
	\item Write a program that reads in a FastA file of the user's    choice, and outputs a numbered list of the titles contained    therein. Ask the user for a number from the menu, make sure that    the number is in range, otherwise ask again.  Display the title and    sequence string of the sequence chosen by the user, then display    the menu again. The user may quit the program by entering 'q' at    the menu prompt.
	\item Write a program that reads in a grid of biomass densities (specified as floats) from a file designated by the user. The grid can be of any dimensions, as long as it is square, and is specified in the file by a number of \textbf{     tab separated} floating point numbers, where tabs indicate column separation, and lines indicate row separation. Ask the user to input a number of hours. Each hour the biomass density in each cell of the grid increases by 5\%. Biomass     densities may not be greater than 1, and excess biomass density is spread evenly amongst the four grid cells to the north, east, south, and west. If those cells are already at density 1 themselves, no growth takes place. The grid does not wrap     around. After the number of hours specified have been calculated, output the grid as a collection of lines of floats indicating the new biomass levels per cell. Sample output might be     
\begin{lstlisting}

        0       1       2       3       4       5       6       7       8       9
0       1.0     1.0     0.5     0.0     0.0     0.0     0.0     0.0     0.0     0.0     
1       1.0     0.8     0.0     0.0     0.0     0.0     0.0     0.0     0.0     0.0
2       0.5     0.0     0.0     0.0     0.0     0.5     0.0     0.0     0.0     0.0
3       0.0     0.0     0.0     0.0     0.8     1.0     0.8     0.0     0.0     0.0
4       0.0     0.0     0.0     0.8     1.0     1.0     1.0     0.5     0.0     0.0
5       0.0     0.0     0.0     0.0     0.8     1.0     0.8     0.0     0.0     0.0
6       0.0     0.0     0.0     0.0     0.0     0.5     0.0     0.0     0.0     0.0
7       0.0     0.0     0.0     0.0     0.0     0.0     0.0     0.0     0.0     0.0
8       0.0     0.0     0.0     0.0     0.0     0.0     0.0     0.0     0.0     0.0
9       0.0     0.0     0.0     0.0     0.0     0.0     0.0     0.0     0.0     0.0
\end{lstlisting}
\end{enumerate}

\hypertarget{week4}{Week 4}
\begin{enumerate}
	\item Write a program that reads a file specified as the first argument on the command line, and outputs it to screen.
	\item Write a program that reads a file specified as the last argument on the command line. If any optional command line arguments are encountered prior to the filename, i.e. more than one command line argument is given, those    arguments specify things to do to the output. If the argument '-b' is encountered, blank lines are not outputted. If the argument '-C' is encountered, the first non white space letter after a period is Capitalised. Any error conditions should be    sent to stderr.
	\item Write a program that accepts as its first command line argument a width in columns, and as its second command line argument a filename. The contents of that file are outputted to screen using lines no longer than width, and not    breaking any words in two.
	\item Write a program to output the CG/AT ratio of all sequences in a FastA file specified on the command line, such that their titles are outputted followed by their GC/AT ratio, one sequence per line. The CG/AT ratio of a sequence    is defined as the sum of occurrences of 'C' and 'G' divided by the sum of occurrences of 'A' and 'T' within that sequence.
	\item Write a program that accepts a single FastA file from the command line and for each sequence in the file, outputs that sequence's title, followed by a sequence of indented lines, one for each coding region found in the sequence,    containing the contents of the coding region. Coding regions subsequent to the first should be in the same reading frame. Coding regions start at start codons ('TAA', 'TAG', 'TGA'), and stop at stop codons ('TTG', 'CTG', 'ATG'). Any match of start    and stop codons defines a coding region.
	\item Describe in a natural language what the regular expression \textbf{Jack Sprat could eat no (fat)|(hat)} matches.
	\item Describe in a natural language what the regular expression \textbf{His wife|girlfriend could eat no lean|bean} matches.
	\item Write down all the possible matches for the regular expression \textbf{All in all Jack's ((wife's)|(girlfriend's) )?dietician was pretty (un)?happy about the situation}.
	\item Write a regular expression that identifies an integer.
	\item Write a regular expression that identifies a float.
	\item Write a regular expression that identifies a coding region.
	\item Write a regular expression that identifies a line containing at least one word float pair, where the word and the float are separated by a comma and any amount of white space before or after the comma, and pairs are separated    from one another in the same way. An example line would be "Height, 0.37 ,$\backslash$tweight,13.2 ,width, 0.54"
\end{enumerate}

\hypertarget{week5}{Week 5}
\begin{enumerate}
	\item Write a program that counts the number of a given character (from the command line) in a given string (also from the command line) recursively.
	\item Write a program that outputs the reverse of a given string (from the command line), by using a recursive reverse function.
	\item Write a non recursive function to sort a list of ten integers generated randomly. Do not use the list.sort() method.
	\item Write a recursive sort function. \textit{Hint:} Write a function that inserts an integer into the correct place in an already sorted list, using recursion. Use this in your sort function.
	\item Write a program that lists all the permutations of a string given on the command line.
	\item A transport company needs to buy five new vehicles for its operation. As they are a start up company, they are very concerned about cash, and would like to spend as little as possible to get the most out of their vehicles. They    want to get as much carrying capacity for as little money as possible, and they cannot afford more than R1 million anyway. Vehicles have the properties: cost, and carrying capacity. Assuming we are supplied the list of vehicle models available from    the command line, each model as an argument specified as '$<$name$>$:$<$cost$>$:$<$capacity$>$', write a program that outputs the optimal combination of five vehicles that give the highest capacity/cost ratio.
\end{enumerate}

\hypertarget{week6}{Week 6}
\begin{enumerate}
	\item Write a vector class for vectors of any size. The class must support vector addition (\_\_sub\_\_), vector subtraction (\_\_sub\_\_), scalar multiplication (\_\_mul\_\_ by integer/float), vector multiplication (\_\_mul\_\_ by vector), unary    negation (\_\_neg\_\_), magnitude/distance (\_\_abs\_\_), scaled magnitude, and dot product operations.
	\item Write a matrix class for matrices of any size. The class must support matrix addition (\_\_add\_\_), matrix subtraction (\_\_sub\_\_), scalar multiplication (\_\_mul\_\_ by integer/float), matrix multiplication (\_\_mul\_\_ by matrix/vector),    unary negation (\_\_neg\_\_), and transposition operations.
\end{enumerate}    [\href{index.html}{Course Outline}]
\\    [\href{solutions}{Solutions}]      Copyright \copyright James Dominy 2007-2008; Released under the \href{http://www.gnu.org/copyleft/fdl.html}{GNU Free Documentation License}
\\\href{intropython.tar.gz}{Download the tarball}

\end{document}
