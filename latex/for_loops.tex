\section{Loops over Lists}
In lesson 8 we introduced the concept of lists. 

A list is a type of variable that contains other elements. For instance we could create a shopping list. 
\begin{lstlisting}
>>> shopping=[`Milk',`Bread',`Eggs',`Motherboard',`Cheese',`Jacket']
\end{lstlisting}

In order to view the individual contents of that list, we could refer to them by index
\begin{lstlisting}
>>> shopping[0]
`Milk'
>>> shopping[3]
`Motherboard'
\end{lstlisting}

 In order to view all the items of our list, we could manually type in all the indexes, but what happens if you add more items later, or if you have 100 items? A better idea would be to use a while loop to change the index. We set the index to 0 to begin with, then while the index is less than length of the list, we print out the item in the list referenced by the index and increment the index.
\begin{lstlisting}
>>> index=0
>>> while (index < len(shopping)):
...     print(shopping[index])
...     index+=1
... 
Milk
Bread
Eggs
Motherboard
Cheese
Jacket
>>> 
\end{lstlisting}

That works, but, once again Python provides us a better tool for the job in the form of the \texttt{for} statement.

The \texttt{for} statement is also a looping statement, like the while statement, except that it loops   over the elements of a list rather than until a boolean expression is false. It looks like this
\begin{lstlisting}
for <element> in <list>:
    statement
    statement
    ...
\end{lstlisting}

Where \texttt{element} is a variable name (not necessarily \texttt{element}),   which may be as yet unassigned (the for statement will assign a value   to it), or may be an already existing variable, in which case a new value   will be assigned to that variable by the for statement. \texttt{list} is an   expression whose value is of type list.

What this does exactly is execute the indented statements \textit{once for each element in the list \texttt{list}}. Each time the statements are executed, the variable in the position of \texttt{element}   is assigned the value of the next element in \texttt{list}, starting with the   first.

So, for example, the following program
\begin{lstlisting}
#a small program to display your shopping list
shopping=[`Milk',`Bread',`Eggs',`Motherboard',`Cheese',`Jacket']
for i in shopping:
    print("Item:",i)
\end{lstlisting}

produces the following output
\begin{lstlisting}
Item: Milk
Item: Bread
Item: Eggs
Item: Motherboard
Item: Cheese
Item: Jacket
\end{lstlisting}

Well, that's pretty cool. Notice how the variable i changes on every iteration (every repeat of the loop). But what if we wanted to number the list too, or have two lists, one with the item and one with the price. What we could do is have a list of indexes, and use the for loop to go through them, printing out both the index and the item in the shopping list that corresponds to that number. We could do that as follows: 
\begin{lstlisting}
#a small program to display a numbered shopping list
indexes=[0,1,2,3,4,5]
shopping=[`Milk',`Bread',`Eggs',`Motherboard',`Cheese',`Jacket']
prices=[`8',`6',`10',`800',`30',`150']
for i in indexes:
    print("Item number",i,"is",shopping[i],"and it costs R",prices[i])
\end{lstlisting}   produces the following output  
\lstset{keywordstyle=\ttfamily}
\begin{lstlisting}
Item number 0 is Milk and it costs R 8
Item number 1 is Bread and it costs R 6
Item number 2 is Eggs and it costs R 10
Item number 3 is Motherboard and it costs R 800
Item number 4 is Cheese and it costs R 30
Item number 5 is Jacket and it costs R 150
\end{lstlisting}
\lstset{keywordstyle=\textbf}
\textit{Note that if you were printing this for your mother, she might not understand why milk is the 0th item and not the 1st, so you might want to \texttt{print(i+1)} so that it seems to start at 1.}

 That works alright. But if we were dealing with larger lists,   like of 19 items, or even thousands of items, we really don't   want to have to type out the list of possible indexes ([0, 1, 2, ...,   19]) every time. Python again comes to the rescue with another built in   function, \texttt{range}.   \texttt{range} returns a list of numbers within a given range, and defined as   follows
\begin{lstlisting}
range([start,] stop [, step])
\end{lstlisting}

What it does is returns a list of integers, starting with the number   given by \texttt{start} (or 0 if \texttt{start} is not given), up to but not including   the number given by \texttt{stop}. If \texttt{step} is given, the list will only   provide every \texttt{step}'th integer in the sequence. Examples probably   illustrate this better ...
\begin{lstlisting}
>>> range(5)
[0, 1, 2, 3, 4]
>>> range(4,10)
[4, 5, 6, 7, 8, 9]
>>> range(-3,3)
[-3, -2, -1, 0, 1, 2]
>>> range(0,10,2)
[0, 2, 4, 6, 8]
>>> range(10,20,3)
[10, 13, 16, 19]
>>>
\end{lstlisting}

So we can change our previous program slightly, as follows
\begin{lstlisting}
#a small program to display a numbered shopping list
shopping=[`Milk',`Bread',`Eggs',`Motherboard',`Cheese',`Jacket']
for i in range(len(shopping)):
    print("Item number",i,"is",shopping[i])
\end{lstlisting}

The \texttt{for} loop isn't only used with lists. The \texttt{for} loop can be used with any iterable structure, i.e. any structure which can be indexed. One such structure is the string, which you'll examine in depth in the next chapter. As has been mentioned previously, the string is basically a list of characters, and hence you can use a \texttt{for} loop to iterate through those characters. So for example to print every letter of a word on a new line:   
\begin{lstlisting}
>>> for i in "Any Word":
...     print(i)
... 
A
n
y
 
W
o
r
d
\end{lstlisting}

 The \texttt{for} loop also works on parts sections of lists, and edited lists, here are some examples:
\begin{lstlisting}
>>> for i in ["Ford","Mercedes","BMW","Audi","Toyota","Nissan"][2:5]:
...     print(i)
... 
BMW
Audi
Toyota
>>> for i in "Experts Exchange"[6:10]:
...     print(i,)
... 
s   E x
>>> for i in ["Animals",3,"blah"]+[2,True,"for you"]:
...     print(i, " | ",)
... 
Animals  |  3  |  blah  |  2  |  True  |  for you  | 
>>> for i in "Hello how are you doing?"[::2]:
...     print(i,)
... 
H l o h w a e y u d i g
\end{lstlisting}

\section{else Clauses in for loops}

\texttt{for} loop statements may also have an \texttt{else} clause, which is   executed when the list is finished, but not when the loop is   terminated by a break statement.
\begin{lstlisting}
for <variable> in <list>:
    <statement>
    [statement]
else:
    <statement>
    [statement]
\end{lstlisting}

\section{Exercises}
\begin{enumerate}
	\item Write a program that asks the user to enter two (whole) numbers and outputs the product. However, do so without using `*' but rather by repeatedly adding.
	\item Write a program that prints out the string ``repeat" 100 times.
	\item Write a program that asks the user for their name, and a    number, then prints out the user's name, with some motivational comment, that many times.
	\item What is an expression, using the range function, that yields a    list of 10 consecutive integers starting at 0?
	\item What is an expression, using the range function, that yields a    list of 100 consecutive integers starting from 1?
	\item What is an expression, using the range function, that yields a    list of 100 consecutive \textit{even} integers starting    from 2?
	\item Write a program that prints the \textit{odd} numbers from 1 to 100 each on    its own line.
	\item The triangular numbers are the numbers 1, 3, 6, 10, 15... The \textit{n}'th triangular number is the sum of all the integers from 1 till \textit{n}. e.g. the 4th triangular number is 1+2+3+4=10. Write a program that accepts an integer value for \textit{n} from the user, and prints the \textit{n}'th triangular number.
	\item Write a program that accepts words from the user, storing them in a list, until the user enters the word ``end". The program must then print out all the words in reverse order.
	\item Modify your answer to question 10 from ``Flow Control:    Conditionals", i.e. print out the numbers from 1 to 10 by which a    user entered number is divisible, to use a for loop instead of    multiple if statements.
	\item Write a program that asks the user for the height of a triangle.    It must then print out a right handed triangle, with the right-angle on the    bottom left, made of asterisks (`*') of a height equal to the    number entered. Example input/output follows ...      
\begin{lstlisting}
Enter triangle height: 3
*
**
***
\end{lstlisting}
\begin{lstlisting}
Enter triangle height: 5
*
**
***
****
*****
\end{lstlisting}
	\item Do the same as above, except now with the right-angle in the top right, and after you have printed the first triangle, you must start all over again until the user enters a blank line for input. Example ...      
\begin{lstlisting}
Enter triangle height: 3
***
 **
  *
Enter triangle height: 5
*****
 ****
  ***
   **
    *
Enter triangle height: 
\end{lstlisting}
	\item Write a program that prints out a `Christmas' tree shape from    asterisks. The program should ask the user for the height of the    tree, in lines, and the tree should have a stalk of two lines    regardless. If the user enters a height, the tree should be printed    out, and the user should be prompted for another height until they    enter a blank line.     
\begin{lstlisting}
Enter tree height: 4
   *
  ***
 *****
*******
   *
   *
Enter tree height: 6
     *
    ***
   *****
  *******
 *********
***********
     *
     *
\end{lstlisting}
\end{enumerate}    